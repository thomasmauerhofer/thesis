\chapter{Introduction}
\label{cha:introduction}

\section{Background}
\label{sec:background}

Background of information retrieval...

\section{Motivation}
\label{sec:Motivation}

\myfig{implicit_vs_explicit}
      {width=1.0\textwidth}
      {Implicit and Explicit Search}
      {Implicit and Explicit Search}
      {fig:implicit_vs_explicit}

\myfig{input_search_areas}
      {width=0.50\textwidth}
      {Input Area and Search Area}
      {Input Area and Search Area}
      {fig:input_search_areas}


Main Question: Is it possible to improve the search result quality while searching for scientific publications through the use of IMRaD structure information?

Subquestion: Does the search result improve for explicit search using queries?

Subquestion: Does the search result improve for implicit search using complete scientific publications?

Subquestion: Does the search result improve if only a single chapter of scientific publications is used for searching?

Subsubquestion: Which chapter has the most influence?

Last - no result until now: How is it possible to find papers releated to a set of publications using clustered information about their similarities in various sections?
