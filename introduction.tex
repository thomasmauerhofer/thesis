\chapter{Introduction}
\label{cha:introduction}

\section{Background}
\label{sec:background}

Background of information retrieval...

\section{Motivation}
\label{sec:Motivation}

\myfig{implicit_vs_explicit}
      {width=1.0\textwidth}
      {Implicit and Explicit Search}
      {Implicit and Explicit Search}
      {fig:implicit_vs_explicit}

\myfig{input_search_areas}
      {width=0.50\textwidth}
      {Input Area and Search Area}
      {Input Area and Search Area}
      {fig:input_search_areas}

How is it possible to improve the search result quality while searching for scientific publications through the use of structural information?


Does the search result improve if only a single chapter is used for searching?
Which chapter has the most influence?
Does it make sense to use aditional information about the chapter in the search step?

RQ 1: How effective is it to search sections which explicitly contain keywords, for structural similarities?

RQ 2: When looking at a single publication, is it more efficent to find releated papers using keywords from the whole document, or only from the background section?

RQ 3: How is it possible to find papers releated to a set of publications using clustered information about their similarities in various sections?
