\chapter{Introduction}
\label{cha:introduction}
2 seiten.

Today a world without internet is unimaginable to many people. It affects almost all areas of our daily life. \Cref{fig:60minutes} shows the user activity of the internet within a single minute in $2018$. In this minute, for example, $3.7$ million search queries were made on Google, $18$ million messages were sent on Whatsapp, or $4.3$ million videos were viewed on YouTube. This suggests that the internet is big, fast-, and constantly changing.

Through this amount of data almost everything can be found on the internet. However a typical user is only interested into a specific piece of information. Therefore ... search engines



/*This descripes the usage of information retrieval where information resources are obtained by the usage of a given piece of information.*/ "Typical user is only interestet into a specific piece of information. therefore... in der regel user der suchbegriffe eingibt. suchmaschienen sind optimiert die relefanten sachen zu finden. page rank paper" Applications which use information retrieval to provide these informations to the user are better known as search engines.

During a single search step the engines use different information channels. The first one is the information what the user explicitly searches for. This can be query words, an image, a document, or any other kind of information the user provides to the engine. Using only this information is called explicit search. But there is also other information available, which was not explicitly provided by the user. This can be for example the location of the user, the date of the image, or structural information about a document. Using this kind of information additional to the explicit information is called implicit search. %Addidtionaly information about the user..can used

Search engines have many different areas in which they are used. One of them is in the field of science, where they are used to simplify the literature research. process ein wenig beschreiben. einfach in bibiothek... modern über suchmaschinen. ich hab ein problem, und wie lös ich.


\myfig{60minutes}
      {width=0.70\textwidth}
      {\textbf{User activity of the Internet in 2018.} The Internet has become a daily companion, and is an indispensable part of life. This graph gives an overview of what happens within the most popular services in a single minute.  Quelle: Lori Lewis (https://www.allaccess.com/merge/archive/28030/2018-update-what-happens-in-an-internet-minute)}
      {User activity of the Internet in 2018.}
      {fig:60minutes}

\section{Motivation}
\label{sec:Motivation}

über meine suchmaschine... was bringt es. Auf imrad eingehen. ein merkmal ist imrad struktur.

\myfig{implicit_vs_explicit}
      {width=1.0\textwidth}
      {Implicit and Explicit Search}
      {Implicit and Explicit Search}
      {fig:implicit_vs_explicit}

\myfig{input_search_areas}
      {width=0.50\textwidth}
      {Input Area and Search Area}
      {Input Area and Search Area}
      {fig:input_search_areas}


\section{Research Questions}
\label{sec:research_questions}

Gegeben aus meiner Motivation geben sich folgende research questions. und dann bei jeder kurz dazu was es bedeutet.


Main Question: Is it possible to improve the search result quality while searching for scientific publications through the use of IMRaD structure information?

Subquestion: Does the search result improve for explicit search using queries?

Subquestion: Does the search result improve for implicit search using complete scientific publications?

Subquestion: Does the search result improve if only a single chapter of scientific publications is used for searching?

Subsubquestion: Which chapter has the most influence?

Last - no result until now: How is it possible to find papers releated to a set of publications using clustered information about their similarities in various sections?
