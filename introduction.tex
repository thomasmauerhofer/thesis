\chapter{Introduction}
\label{cha:introduction}

Today a world without internet is unimaginable to many people. It affects almost all areas of our daily life. \Cref{fig:60minutes} showes the user activity of the internet within a single minute in 2018. In this minute, for example, 3.7 million search queries were made on Google, 18 million messages were sent on Whatsapp, or 4.3 million videos were viewed on YouTube. This suggests that the internet is big, fast-, and constantly changing. Through this amount of data almost everything can be found on the internet, but not every piece of information has the same relevance for the user. This descripes the usage of information retrieval where information resources are obtained by the usage of a given piece of information. Applications which use information retrieval to provide these informations to the user are better known as search engines. Search engines are used in many different areas, such as Google for the web,  (1, or 2 more examples) Google Scholar for scientific writings.

Sprung zu search engines for scientific writings. (wichtig weil A: man muss abgrenzen, B: Context der arbeit, C: Methaarbeiten (wo üblich über teilbereich))

\myfig{60minutes}
      {width=0.70\textwidth}
      {\textbf{User activity of the Internet in 2018.} In a single minute What happens in an Internet Minute  Quelle: Lori Lewis (https://www.allaccess.com/merge/archive/28030/2018-update-what-happens-in-an-internet-minute)}
      {User activity of the Internet in 2018.}
      {fig:60minutes}

\section{Background}
\label{sec:background}

Background of information retrieval...

\section{Motivation}
\label{sec:Motivation}

\myfig{implicit_vs_explicit}
      {width=1.0\textwidth}
      {Implicit and Explicit Search}
      {Implicit and Explicit Search}
      {fig:implicit_vs_explicit}

\myfig{input_search_areas}
      {width=0.50\textwidth}
      {Input Area and Search Area}
      {Input Area and Search Area}
      {fig:input_search_areas}


Main Question: Is it possible to improve the search result quality while searching for scientific publications through the use of IMRaD structure information?

Subquestion: Does the search result improve for explicit search using queries?

Subquestion: Does the search result improve for implicit search using complete scientific publications?

Subquestion: Does the search result improve if only a single chapter of scientific publications is used for searching?

Subsubquestion: Which chapter has the most influence?

Last - no result until now: How is it possible to find papers releated to a set of publications using clustered information about their similarities in various sections?
