\newcommand{\mypapersize}{A4}
%% e.g., "A4", "letter", "legal", "executive", ...
%% The size of the paper of the resulting PDF file.

\newcommand{\mylaterality}{oneside}
%% "oneside" or "twoside"
%% Either you are creating a document which is printed on both, left pages
%% and right pages (twoside) or you create a document which is printed
%% on right pages only (oneside).

\newcommand{\mydraft}{false}
%% "true" or "false"
%% Use draft mode? If true, included graphics are replaced by empty
%% rectangles (of same size) and overfull boxes (in margin space) are
%% marked with black box (-> easy to spot!)

\newcommand{\myparskip}{full}
%% e.g., "no", "full", "half", ...
%% How to separate paragraphs: indention ("no") or spacing ("half",
%% "full", ...).

\newcommand{\myBCOR}{0mm}
%% Inner binding correction. This value depends on the method which is
%% being used to bind your printed result. Some techniques do not
%% require a binding correction at all ("0mm"), other require for
%% example "5mm". Refer to KOMA script documentation for a detailed
%% explanation what a binding correction is and how to measure it.

\newcommand{\myfontsize}{12pt}
%% e.g., 10pt, 11pt, 12pt
%% The font size of the main text in pt (points).

\newcommand{\mylinespread}{1.0}
%% e.g., 1.0, 1.5, 2.0
%% Line spacing in %/100. For example 1.5 means 150% of the usual line
%% spacing. Please use with caution: 100% ("1.0") is fine because the
%% font was designed for it.

\newcommand{\mylanguage}{ngerman,american}
%% NOTE: The *last* language is the active one!
%% See babel documentation for further details.

%% BibLaTeX-settings: (see biblatex reference for further description)
\newcommand{\mybiblatexstyle}{alphabetic}
%% e.g., "alphabetic", "authoryear", ...
%% The biblatex style which is being used for referencing. See
%% biblatex documentation for further details and more values.
%%
%% CAUTION: if you change the style, please check for (in)compatible
%%          "biblatex" package options in the file
%%          "template/preamble.tex"! For example: "alphabetic" does
%%          not have an option "dashed=..." and causes an error if it
%%          does not get removed from the list of options.

\newcommand{\mybiblatexdashed}{false}  %% "true" or "false"
%% If true: replace recurring reference authors with a dash.

\newcommand{\mybiblatexbackref}{true}  %% "true" or "false"
%% If true: create backward links from reference to citations.

\newcommand{\mybiblatexfile}{references-biblatex.bib}
%% Name of the biblatex file that holds the references.

\newcommand{\mydispositioncolor}{0,0,0}
%% e.g., "30,103,182" (blue/turquois), "0,0,0" (black), ...
%% Color of the headings and so forth in RGB (red,green,blue) values.
%% NOTE: if you are using "0,0,0" for black, printers might still
%%       recognize pages as color pages. In case this is a problem
%%       (paying for color print-outs vs. paying for b/w-printouts)
%%       please edit file "template/preamble.tex" and change
%%       "\definecolor{DispositionColor}{RGB}{\mydispositioncolor}"
%%       to "\definecolor{DispositionColor}{gray}{0}" and thus
%%       overwriting the value of \mydispositioncolor above.

\newcommand{\mycolorlinks}{true}  %% "true" or "false"
%% Enables or disables colored links (hyperref package).

\newcommand{\mytitlepage}{template/title_Thesis_TU_Graz}
%% Your own or one of following pre-defined title pages:
%% "template/title_plain_maketitle": simple maketitle page
%% "template/title_Diplomarbeit_KF_Uni_Graz.tex": fancy (german) title page for KF Uni Graz
%% "template/title_Thesis_TU_Graz":
%%             titlepage for Graz University of Technology (correct
%%             (old?) Corporate Design) by Karl Voit (2012)
%% "template/title_Thesis_TU_Graz_-_kazemakase":
%%             titlepage for Graz University of Technology
%%             (correct new Corporate Design) by kazemakase (2013):
%%             see https://github.com/novoid/LaTeX-KOMA-template/issues/5
%% "template/title_VWA": titlepage for Vorwissenschaftliche Arbeit

\newcommand{\mytodonotesoptions}{disable}
%% e.g., "" (empty), "disable", ...
%% Options for the todonotes-package. If "disable", all todonotes will
%% be hidden (including listoftodos).

%% Load main settings for document preamble:
\input{template/preamble}%% DO NOT REMOVE THIS LINE!

\setboolean{myaddcolophon}{true}  %% "true" or "false"
%% If set to "true": a colophon (with notes about this document
%% template, LaTeX, ...) is added after the title page.
%% Please do not set to "false" without a good reason. The colophon
%% helps your readers to get in touch with LaTeX and to find this template.

\setboolean{myaddlistoftodos}{true}  %% "true" or "false"
%% If set to "true": the current list of open todos is added after the
%% table of contents. If \mytodonotesoptions is set to "disable", no
%% list of todos is added, independent of this setting here.

\setboolean{english_affidavit}{true}  %% "true" or "false"
%% If set to "true": the language of the statutory declaration text is set to
%% English, otherwise it is in German.


%% ========================================================================
%%%% Document metadata
%% ========================================================================

%% general metadata:
\newcommand{\myauthor}{Thomas Mauerhofer}  %% also used for PDF metadata (hyperref)
\newcommand{\myauthorwithexistingtitles}{\myauthor{}, BSc}  %% including
                                %% university degree already held
                                %% (BSc, MSc, ...)
\newcommand{\mytitle}{Using IMRaD Structure Features in Information Retrieval Raking Functions}  %% also used for PDF metadata (hyperref)
\newcommand{\mysubject}{Master's Thesis}  %% also used for PDF metadata (hyperref)
\newcommand{\mykeywords}{Information Retrieval; Structured Text Retrieval; IMRaD; Term Frequency; TF-IDF; Ranked Boolean Retrieval; Okapi BM$25$; Divergence from Randomness}  %% also used for PDF metadata (hyperref)

%% this information is used only for generating the title page:
\newcommand{\myworktitle}{Master's Thesis}  %% official type of work like ``Master theses''
\newcommand{\mygrade}{Diplom-Ingenieur} %% title you are getting with this work like ``Master of ...''
\newcommand{\mystudy}{Computer Science} %% your study like ``Arts''
\newcommand{\mydegreeprogramme}{Master's degree programme: \mystudy} %% Master's or PhD degree programme
\newcommand{\myuniversity}{Graz University of Technology} %% your university/school
\newcommand{\myinstitute}{Institute of Interactive Systems and Data Science} %% affiliation
\newcommand{\myinstitutehead}{Univ.-Prof.~Dipl.-Inf.~Dr.~Stefanie Lindstaedt} %% head of institute
\newcommand{\mysupervisor}{Ass.Prof.~Dipl.-Ing.~Dr.techn.~Roman Kern} %% your supervisor
\newcommand{\myhomestreet}{Neufeldweg~75} %% your home street (with house number)
\newcommand{\myhometown}{Graz} %% your home town
\newcommand{\myhomepostalnumber}{8010} %% your postal number of home town
\newcommand{\mysubmissionmonth}{September} %% month you are handing in
\newcommand{\mysubmissionyear}{2020} %% year you are handing in
\newcommand{\mysubmissiontown}{\myhometown} %% town of handing in (or \myhometown)

%% additional information for generic_documentation title page
\newcommand{\myid}{1031957} %% Matrikelnummer
\newcommand{\mylecture}{Master's Thesis} %%


%% ========================================================================
%%%% MISC command definitions
%% ========================================================================
\input{template/mycommands}

%% ========================================================================
%%%% Typographic settings
%% ========================================================================
\input{template/typographic_settings}


%% ========================================================================
%%%% MISC usepackages
%% ========================================================================

%% ... it's OK to put here your own usepackage commands ...
\usepackage{color, colortbl}
\usepackage{multirow}
\usepackage{booktabs}
\usepackage{subcaption}
\usepackage{blkarray}
\usepackage{pgfplots}
\usepackage{pgfplotstable}
\usepackage{floatpag}
\usepackage{listings}

\definecolor{maroon}{rgb}{0.5,0,0}
\definecolor{darkgreen}{rgb}{0,0.5,0}
\lstdefinelanguage{XML}
{
  basicstyle=\ttfamily,
  morestring=[s]{"}{"},
  morecomment=[s]{?}{?},
  morecomment=[s]{!--}{--},
  commentstyle=\color{darkgreen},
  moredelim=[s][\color{black}]{>}{<},
  moredelim=[s][\color{red}]{\ }{=},
  stringstyle=\color{blue},
  identifierstyle=\color{maroon}
}

% headmap setup
\pgfplotstableset{
    /color cells/min/.initial=0,
    /color cells/max/.initial=1000,
    /color cells/textcolor/.initial=,
    %
    % Usage: ‘color cells={min=<value which is mapped to lowest color>,
    %   max = <value which is mapped to largest>}
    color cells/.code={%
        \pgfqkeys{/color cells}{#1}%
        \pgfkeysalso{%
            postproc cell content/.code={%
                %
                \begingroup
                %
                % acquire the value before any number printer changed
                % it:
                \pgfkeysgetvalue{/pgfplots/table/@preprocessed cell content}\value
\ifx\value\empty
\endgroup
\else
                \pgfmathfloatparsenumber{\value}%
                \pgfmathfloattofixed{\pgfmathresult}%
                \let\value=\pgfmathresult
                %
                % map that value:
                \pgfplotscolormapaccess
                    [\pgfkeysvalueof{/color cells/min}:\pgfkeysvalueof{/color cells/max}]%
                    {\value}%
                    {\pgfkeysvalueof{/pgfplots/colormap name}}%
                % now, \pgfmathresult contains {<R>,<G>,<B>}
                %
                % acquire the value AFTER any preprocessor or
                % typesetter (like number printer) worked on it:
                \pgfkeysgetvalue{/pgfplots/table/@cell content}\typesetvalue
                \pgfkeysgetvalue{/color cells/textcolor}\textcolorvalue
                %
                % tex-expansion control
                % see http://tex.stackexchange.com/questions/12668/where-do-i-start-latex-programming/27589#27589
                \toks0=\expandafter{\typesetvalue}%
                \xdef\temp{%
                    \noexpand\pgfkeysalso{%
                        @cell content={%
                            \noexpand\cellcolor[rgb]{\pgfmathresult}%
                            \noexpand\definecolor{mapped color}{rgb}{\pgfmathresult}%
                            \ifx\textcolorvalue\empty
                            \else
                                \noexpand\color{\textcolorvalue}%
                            \fi
                            \the\toks0 %
                        }%
                    }%
                }%
                \endgroup
                \temp
\fi
            }%
        }%
    }
}


%% ========================================================================
%%%% MISC self-defined commands and settings
%% ========================================================================

%% ... it's OK to put here your own newcommand/newenvironment-definitions ...
\definecolor{lightblue}{rgb}{0.93,0.95,1.0}



\newcommand{\myLaT}{\LaTeX{}@TUG\xspace} %% LaTeX@TUG text "logo"

\hyphenation{ex-am-ple hy-phen-ate}  %% in order to use German umlauts
%% here (Ver-\"of-fent-li-chung), you have to check for
%% activated \usepackage[T1]{fontenc} in the preamble

%% override default language of babel: (be sure to know, what you're
%% doing here)
%\selectlanguage{american}
%\selectlanguage{ngerman}

%% ========================================================================
%%%% Templates
%% ========================================================================

%% template for inserting figures:
% \myfig{}%% filename
%       {}%% width/height
%       {}%% caption
%       {}%% optional (short) caption for list of figures
%       {fig:}%% label

%% acronyms in small caps: \myacro{UNESCO}


\input{template/pdf_settings}  %% should be *last* definitions in preamble!
%% ========================================================================
%%%% begin{document}
%% ========================================================================
\begin{document}

\frontmatter                    %% KOMA: roman page numbers and such; only available in scrbook

\input{colophon}                %% defines information about editor, LaTeX, font, ...

%% Choose your desired title page:
\input{\mytitlepage}            %% include title page


\input{template/declaration_TU_Graz}  %% Statutory Declaration
% \input{thanks}                %% this is a suggestion: you have to create this file on demand
% \input{foreword}              %% this is a suggestion: you have to create this file on demand


%% include the abstract without chapter number but include it on table of contents:
\cleardoublepage
\phantomsection
\addcontentsline{toc}{chapter}{Abstract}
\chapter*{Abstract}
\label{cha:abstract}

Today the internet is growing fast as users generate a lot of data. Therefore, finding relevant information is getting more and more time-consuming. This happens as the time required to find a piece of information, and the number of information sources that need to be searched increases. Search engines filter data, and reduce the time required to find relevant information. We focus on the field of science where they help to simplify literature search. An advantage of scientific articles is that they have a common structure to increase the readability. This structure is known is IMRaD (Introduction, Method, Results and Discussion). We tackle the question whether it is possible to improve the search result quality while searching for scientific works by using IMRaD structure information. We used several state-of-the-art ranking algorithms, and compare them against each other in our experiments. Our results show that the importance of IMRaD chapter features depends on the complexity of the query. Finally, we focus on structured text retrieval and the influence of single chapters on the search result.


\textbf{Keywords:} \mykeywords{}              %% Abstract


\tableofcontents                %% this produces the table of contents - you might have guessed :-)

\listoffigures
\listoftables

%% if myaddlistoftodos is set to "true", the current list of open todos is added:
\ifthenelse{\boolean{myaddlistoftodos}}{
  \newpage\listoftodos          %% handy if you are using todonotes with \todo{}
}{}                             %% with todonotes-package option "disable" you can get rid of any todo in the output

\mainmatter                     %% KOMA: marks main part using arabic page numbers and such; only available in scrbook

%% include tex file chapters:
\chapter{Introduction}
\label{cha:introduction}

\section{Background}
\label{sec:background}

Background of information retrieval...

\section{Motivation}
\label{sec:Motivation}

\myfig{implicit_vs_explicit}
      {width=1.0\textwidth}
      {Implicit and Explicit Search}
      {Implicit and Explicit Search}
      {fig:implicit_vs_explicit}

\myfig{input_search_areas}
      {width=0.50\textwidth}
      {Input Area and Search Area}
      {Input Area and Search Area}
      {fig:input_search_areas}


Main Question: Is it possible to improve the search result quality while searching for scientific publications through the use of IMRaD structure information?

Subquestion: Does the search result improve for explicit search using queries?

Subquestion: Does the search result improve for implicit search using complete scientific publications?

Subquestion: Does the search result improve if only a single chapter of scientific publications is used for searching?

Subsubquestion: Which chapter has the most influence?

Last - no result until now: How is it possible to find papers releated to a set of publications using clustered information about their similarities in various sections?

\chapter{Related Work}
\label{cha:related_work}

\section{Information Retrieval Models}
\label{sec:information_retrieval_models}

Creating an information retrieval system is a complex process that has to be planned accordingly. To reach this goal models are used as a base, where the whole system is sketched. The model generation consists of two tasks. First, design a framework which represents the documents and the user queries. Second, create a ranking function, which generates a numeric rank for each document based on a query. Afterwards these ranks are used by the system to sort the documents.

One of the most common retrieval approaches is retrieval based on index terms. In this context an index term is a keyword, which appears in the document collection of the framework. This approach can be implemented efficiently as query words can be used as index terms with limited transformations. For example a user is interested in cooking, and searches for "Austrian dishes". The query words "Austrian", and "dishes" can directly used to search through the document collection since they do not need any transformation.

In general, information retrieval models consists of four parts. Ribeiro-Neto and Baeza-Yates ~\cite{ModernInvormationRetrieval1999} define them as a quadruple $[\textbf{D}, \textbf{Q}, \mathcal{F}, \mathcal{R}(q_i, d_j)]$, where:

\begin{enumerate}
  \item \textbf{D} is a set composed of logical views of documents in a collection.
  \item \textbf{Q} is a set composed of logical views of the user information needs. Such representations are called queries.
  \item $\mathcal{F}$ is a framework for modeling document representations, queries, and their relationships.
  \item $\mathcal{R}(q_i, d_j)$ is a raking function that associates a real number with a query representation $q_i \in \textbf{Q}$ and a document representation $d_j \in \textbf{D}$. The ranking function generates an order over all documents \textbf{D} with respect to a query $q_i$.
\end{enumerate}

Hence, the model is used to define the framework $\mathcal{F}$ and the ranking function $\mathcal{R}(q_i, d_j)$. For example, for textual documents the document representation is a set of all terms within the document. To keep the collection smaller without losing any information stop words should be removed in a preprocessing step. The set of index terms within a document collection is called vocabulary. According to our document representation the query representation is a set of all terms within the query. There can also be an additional preprocessing step in the query creation. An example for such an preprocessing step would be synonyms which are added to the query set.

After the design of the framework, a ranking function is created. It should be constructed in a way that it fits to the requirements of the user. This means for a given query, the ranking function determines a numeric rank to each document in the collection, which represents the relevance for the user. For example, the ranking function counts how many query terms appear in the term set of a document.

Another example is to use term frequency as ranking function. Term frequency itself denotes how often a term occurs in a document. To be able to use it, the document representation is adapted from a set with all terms to a bag of words. In a bag of words each term is represented as a pair of term and term frequency. The ranking function sums the frequencies over all query terms. To be able to compare the documents using the term frequency, the ranks are normalized.

Information retrieval is used in several fields where the underlying models have to fulfill different requirements. Therefore, they are separated into text-based models, link-based models, and multimedia objects-based models. Furthermore, text-based models can be categorized in unstructured, and semi-structured text models. Unstructured text models are used for text documents where the content is represented as sequence of words. Semi-structured text models contain structure such as title, sections, paragraphs, in addition to unstructured text.

The web is rapidly growing, and as a consequence has a huge number of web pages (i.e. documents). Therefore, additional information has to be leveraged as well. This means that the content of documents, and furthermore the links between those documents are take into account. Models which use those additional link information are called Link-based models where PageRank ~\cite{brin1998anatomy} and Hyperlink-Induced Topic Search ~\cite{kleinberg1999authoritative} are important parts of the models.

Information retrieval for multimedia objects differs according their underlying data from the first $2$ types. For example, when thinking on a image it can be seen as a matrix of color values. Detecting similarities between images requires the calculation of more complex features, such as shapes. The representation of the query has to be adapted as well. The user can use words, or use images to define a query. One of the simplest forms of multimedia-based retrieval is image retrieval. Audio and video retrieval are more complicated since there is also a time value which have to be taken into account.

\section{Unstructured Text Retrieval}
\label{sec:unstructured_text_Retrieval}

In unstructured text retrieval, documents can be seen as sequence of words. The $3$ classical models are boolean-, vector-, and probabilistic model. First, in the boolean model, documents and queries are represented as sets. Terms are stitched together with boolean operators to formulize user queries. Second, in the vector model, documents and queries are represented as a vector in a t-dimensional space. The size of t is defined by the number of words in the vocabulary of the collection. Third, in the probabilistic model, documents and queries are represented based on probability theory. Specifically by estimating the probability of a term appearing in a relevant document. Gudivada et al. ~\cite{gudivada1997} advice in their work to denote boolean models as set theoretic, vector models as algebraic, and probabilistic models as probabilistic.

\subsection{The Boolean Model}
\label{sec:the_boolean_model}

The boolean model is a well-known information retrieval model in the area of unstructured text retrieval. It was proposed as a paradigm for accessing large-scale systems since the $1950$s ~\cite{Melucci2009}. The model uses boolean operators and set theory to find relevant documents.

\myfig{boolean_model}
      {width=0.65\textwidth}
      {\textbf{Example query in the boolean model with $3$ terms.} For the boolean model documents in the collection are represented as sets of terms. In this example the vocabulary of the document collection is given by V = \{$t_1$, $t_2$, $t_3$\}. Furthermore, documents can be separated according to the terms they are containing. Given a query $q=t_1 \wedge (t_2 \vee \neg t_3)$ all documents which satisfy this query are marked with an green hook. This means that they are relevant for the user. All other documents which does not satisfy the query are marked with a red cross.}
      {Example query in the boolean model with $3$ terms.}
      {fig:boolean_model}

The classic boolean model can only decide if a document is relevant for the user, or not. It does not provide a rank, which is used to sort the documents. Salton et al. introduce in their work ~\cite{Salton-Extended-1983} an extension where documents are sorted according their relevance.

Index terms are combined with the $3$ boolean operators NOT($\neg$), AND($\wedge$), OR($\vee$) to formulize user queries. The disjunctive normal form of the query shows which areas of the sets are relevant. For example, for query $q=t_1 \wedge (t_2 \vee \neg t_3)$, and vocabulary V = \{$t_1$, $t_2$, $t_3$\}, $q_{DNF}$ is:
\begin{equation}
q_{DNF} = (t_1 \wedge t_2 \wedge t_3) \vee (t_1 \wedge t_2 \wedge \neg t_3) \vee (t_1 \wedge \neg t_2 \wedge \neg t_3)
\end{equation}
This representation of the query highlight that $3$ areas are relevant for the user. First, all $3$ query terms occur. Second, the first and the second term occur, but not the third. Finally, the first term occurs, but not the second and the third. \Cref{fig:boolean_model} displays the example query represented in a Venn diagram, where the $3$ areas can be seen in a graphical representation.

The boolean model works also if not all terms of the vocabulary are part of the user query. Considering a vocabulary V = \{$t_1$, $t_2$, $t_3$, $t_4$\}, and the previous example query, the disjunctive normal form is:
\begin{equation}
  \begin{aligned}
    q_{DNF} = &(t_1 \wedge t_2 \wedge t_3 \wedge \neg t_4) \vee (t_1 \wedge t_2 \wedge t_3 \wedge t_4) \vee \\
              &(t_1 \wedge t_2 \wedge \neg t_3 \wedge \neg t_4) \vee (t_1 \wedge t_2 \wedge \neg t_3 \wedge t_4) \vee \\
              &(t_1 \wedge \neg t_2 \wedge \neg t_3 \wedge \neg t_4) \vee (t_1 \wedge \neg t_2 \wedge \neg t_3 \wedge t_4)
  \end{aligned}
\end{equation}
The last term (i.e., $t_4$), which is not part of the query, is also considered in the disjunctive normal form. It is added once as present, and once as absent to the other $3$ terms.

The main advantages of the boolean model are the clean formalism, and its simplicity ~\cite{ModernInvormationRetrieval1999}. These advantages comes with the usage of binary operators, and binary index term weighing. One of the main disadvantages is the exact matching of documents. This means that a document can only be relevant, or not relevant to the user, without any ranking. As a result, users receive too few or too many documents.

\subsection{The Vector Space Model}
\label{sec:the_vector_space_model}

The vector space model was introduced by Salton et al.~\cite{salton75vsm}. In the model documents and queries are represented as vectors in an t-dimensional space. The size of t is defined by the number of words in the vocabulary of the collection.

The vector space model is more sophisticated than the boolean model, since it contains partial matching. Partial matching means, that a degree of similarity between user queries and the documents in the system are calculated. To accomplish this, non-binary weights are used in combination with index terms.

The idea of non-binary weights is based on the assumption that some index terms are more important than others to describe the content of a document. The calculation of such term weights is a challenging task, since they have to reflect the subjective expectation of a user. As a result terms that appear in a few documents have a higher weight than terms that occur in many documents. For a example a collection consists of $3$ documents, given by $D_1 =$ \{"cooking", "appetizer"\}, $D_2 =$ \{"cooking", "main", "dish"\}, $D_3 =$ \{"cooking", "dessert"\}. A user is interested to create a dessert. Therefore he searches for "cooking dessert". The first query word "cooking" is part of each document. This means it is not very useful to define the users requirements. The second query word "dessert" is only part of one document. %mehr aussagekraft, und dadurch höher gewichtet...



%term weighing

Ribeiro-Neto and Baeza-Yates ~\cite{ModernInvormationRetrieval1999} define the document representation $d_j$, and the query representation $q$ as:
\begin{align}
  \vec{d_j} & = (w_{1, j}, w_{2, j}, \dots, w_{t, j}) \\
  \vec{q} & = (w_{1, q}, w_{2, q}, \dots, w_{t, q})
\end{align}
aaaa bbbb ccccc
% Formeln genau beschreiben. Beispiel mit 3 Wörtern.. one basic example to understand with tf .. überschwingen auf tf-idf....
% Sortierung der documente...

\subsubsection{Term Frequency-Inverse Document Frequency Model}
\label{sec:tfidf}

% also Tf-idf Model in the same paper defined ~\cite{salton75vsm}
% concept of inverse document frequency was introduced by jones72astatistical

Describe basics like term frequency, document frequency...

\begin{equation}
  \begin{split}
    \text{idf}_t & = log \frac{N}{df_t} \\
    \text{tf-idf}_{t, d} & = \text{tf}_{t, d} \cdot \text{idf}_t \\
    \text{Score}(q, d) & = \sum_{t \in q}\text{tf-idf}_{t, d}
  \end{split}
\end{equation}

\subsection{The Probabilistic Model}
\label{sec:the_probabilistic_model}

General about probabilistic model.

\subsubsection{Okapi BM25}
\label{sec:okapi_bm25}
as described in ~\cite{manning2008} page 214 and in ~\cite{ModernInvormationRetrieval1999} page 105
\begin{equation}
  \begin{split}
    \text{idf}_t & = log \frac{N - \text{df}_t + 0.5}{\text{df}_t + 0.5} \\
    \text{bm25}_{t, d} & = \text{idf}_t \cdot \frac{\text{tf}_{t, d} \cdot (k_1 + 1)}{\text{tf}_{t, d} + k_1 \cdot \bigl(1 - b \cdot \frac{|G|}{\text{avgdl}}\bigr)}  \\
    \text{Score}(q, d) & = \sum_{t \in q}\text{bm25}_{t, d}
  \end{split}
\end{equation}


\subsubsection{Divergence from Randomness}
\label{sec:divergence_from_randomness}

as described in ~\cite{ModernInvormationRetrieval1999} page 113
\begin{equation}
  w_{i, j} = (- \log P(k_i | C)) \cdot (1 - P(k_i | d_j))
\end{equation}
\begin{equation}
  \text{Score}(d_j, q) = \sum_{k_i \in q} f_{i, q} \cdot w_{i, j}
\end{equation}
\begin{equation}
  F_i = \sum_j f_{i, j}
\end{equation}
\begin{equation}
  P(k_i | C) = \binom{F_i}{f_{i, j}}p^{f_{i, j}} \cdot (1 - p)^{F_i - f_{i, j}}
\end{equation}
\begin{equation}
  \lambda_i = p \cdot F_i
\end{equation}
\begin{equation}
  P(k_i | C) = \frac{e^{-\lambda_i}\lambda_i^{f_{i, j}}}{f_{i, j}!}
\end{equation}
\begin{equation}
  \begin{split}
    - \log P(k_i | C) & = -\log\Biggl(\frac{e^{-\lambda_i}\lambda_i^{f_{i, j}}}{f_{i, j}!}\Biggr) \\
    & \approx -f_{i,j} \log \lambda_i + \lambda_i \log e + log(f_{i,j}!) \\
    & \approx f_{i, j} \log \Bigl( \frac{f_{i, j}}{\lambda_i} \Bigr) + \Bigl( \lambda_i + \frac{1}{12 f_{i, j} + 1} - f_{i, j}\Bigr) \log e + \frac{1}{2} \log(2 \pi f_{i, j})
  \end{split}
\end{equation}
\begin{equation}
  1 - P(k_i | d_j) = \frac{1}{f_{i, j} + 1}
\end{equation}
\begin{equation}
  f^{\prime}_{i, j} = f_{i, j} \cdot \frac{avgdl}{len(d_j)}
\end{equation}

\section{Structured Text Retrieval}
\label{sec:structured_text_Retrieval}

Write about structuring, early text retrieval, xml retrieval

\subsubsection{Ranked Boolean Retrieval}
\label{sec:ranked_boolean_retrieval}

Described in ~\cite{manning2008} page 103

\begin{equation}
  \sum_{i = 1}^{l}g_i s_i
\end{equation}

\section{Document Preprocessing}
\label{sec:document_preprocessing}

~\cite{ModernInvormationRetrieval1999} page 223

\subsection{Stop Words}
\label{subsec:stop_words}

What are stopwords, and describe common techniques. In ~\cite{Vijayarani2015} and references

\subsection{Stemming}
\label{subsec:stemming}

Describe stemming and common stemming techniques. In ~\cite{Vijayarani2015} and references

\section{Text Similarities}
\label{sec:text_similarities}

Describe Text Similarities and common techniques. ~\cite{ModernInvormationRetrieval1999} page 222

\section{IMRaD Structure}
\label{sec:imrad_structure}

general about IMRaD in scientific writing: ~\cite{robert1989}, chapter distribution analysis: ~\cite{bertin2013} important: ~\cite{Sollaci-The-2004}

\section{Evaluation of Ranking Algorithms}
\label{sec:evaluation_of_ranking_algorithms}

as described in ~\cite{manning2008} page 147 and in ~\cite{ModernInvormationRetrieval1999} page 140

\myfig{precision_recall}
      {width=0.50\textwidth}
      {Precision and Recall}
      {Precision and Recall}
      {fig:precision_recall}

\begin{equation}
  P = \frac{\text{\# relevant items retrieved}}{\text{\# retrieved items}} = \frac{TP}{TP + FP} = P(\text{relevant} | \text{retrieved})
\end{equation}

\myfig{map}
      {width=1.00\textwidth}
      {Example for the precision of a search result}
      {Example for the precision of a search result}
      {fig:map}

\begin{equation}
  \text{MAP}(Q) = \frac{1}{|Q|}\sum_{j = 1}^{|Q|} \frac{1}{m_j}\sum_{k = 1}^{m_j}\text{Precision}(R_{jk})
\end{equation}

\chapter{Materials and Method}
\label{cha:materials_and_method}

\section{Dataset}
\label{sec:dataset}

\subsection{Generation}
\label{subsec:generation}

We created our dataset from approximately $3,000$ scientific articles in PDF format. An important point was that these articles come from different scientific fields.

We used a text mining pre-processing technique as introduced by Vijayarani et al.~\cite{Vijayarani2015} to separate the text from the PDFs and add additional information. This technique consists of three key steps, which are called extraction, stopword removal, and stemming. First, we used a framework described in ~\cite{KlampflGJK14} to separated the article structure and the raw text from the PDF. Afterwards the stopwords are removed, and the remaining terms of the raw text are stemmed.

\begin{table}[!b]
  \caption[Mapping of the Section Titles to IMRaD-Types]{\textbf{Mapping of the Section Titles to IMRaD-Types.} In this Table we show which section titles was used to generate the IMRaD structure information. Additionally, we show the relation between titles and how often they occurs in the used dataset.}
  \centering
  \begin{tabular}{C{2.6cm} C{2.7cm} C{1.5cm} C{1.5cm} C{1.5cm} C{1.5cm}}
    \toprule
    & \multicolumn{3}{c}{\textbf{per Section}} & \multicolumn{2}{c}{\textbf{Overall}} \\
    \textbf{IMRaD Type} & \textbf{Section Title} & \textbf{\# Paper} & \textbf{Percent} & \textbf{\# Paper} & \textbf{Percent} \\ \midrule
    Introduction & Introduction & $822$ & $100\%$ & $822$ & $100\%$ \\ \midrule
    Related Work & Related Work & $465$ & $56.57\%$ & $465$ & $56.57\%$ \\ \midrule
    \multirow{3}{*}{Methods} & Method & $97$ & $11.8\%$ & \multirow{3}{*}{$312$} & \multirow{3}{*}{$37.96\%$} \\
    & Model & 134 & $16.3\%$ \\
    & Approach & 81 & $9.85\%$ \\ \midrule
    \multirow{3}{*}{Result} & Experience & $396$ & $48.18\%$ & \multirow{3}{*}{$687$} & \multirow{3}{*}{$83.58\%$} \\
    & Result & 163 & $19.83\%$ \\
    & Evaluation & 128 & $15.57\%$ \\ \midrule
    \multirow{3}{*}{Discussion} & Conclusion & $581$ & $70.68\%$ & \multirow{3}{*}{$773$} & \multirow{3}{*}{$94.04\%$} \\
    & Discussion & 179 & $21.78\%$ \\
    & Future Work & 13 & $1.58\%$ \\
    \bottomrule
  \end{tabular}
  \label{tbl:mapping_section_names}
\end{table}

One additional information we had to add was IMRaD structure. The IMRaD structure maps Related Work to be part of the Introduction. Since most of our papers had an own section titled Related Work we introduce an additional type for it. We were able to classify the IMRaD structure with simple keyword detection in the section titles. \Cref{tbl:mapping_section_names} shows the mapping between IMRaD-Types and these titles. Because it was not really possible to identify Method sections by using the section titles we used information about their position in the article. We manage that by using the Related Work as upper bound, and the Result section as lower bound. We classify all sections between these two bounds as a Method section. If the Related Work section was not available, we use the Introduction section as upper bound, or if the Result section was not available, we use the Discussion section as lower bound. If one of the two bounds could not be set, we discarded the scientific article. Note that, chapters can have several IMRaD-Types. For example, if a section was titled \textit{"Results and Discussion"} it belongs to the types result and discussion.

Another additional detail we had to add were links between the scientific articles. For this we performed a semi-automated annotation. Thereby similarities about the references of an article and the titles of all other articles are compared, and if they exceed a given threshold a recommendation to create a link between the two was given.

We created each data record in such a way that it can be transferred directly to the database schema shown in \Cref{fig:database_schema}. To reduce noise during the evaluation we removed all articles without any connection to other articles.

Finally, we have $821$ scientific articles in our dataset. This are only $27$ percent of the initial set. This small number is due to the environment and the used scientific articles. One major problem was that the framework used to separate the structure had issues with documents that were not created with latex.

\myfig{database_schema}
      {width=0.70\textwidth}
      {Database Schema for the used Dataset}
      {Database Schema for the used Dataset}
      {fig:database_schema}

\subsection{Structure of a Scientific Article}
\label{sec:structure_scientific_article}

We designed a database schema, which corresponds to the structure of a scientific article. The table "scientific article" in the schema (see \Cref{fig:database_schema}) can be seen as the root node for each database entry.

The "author text" attribute refers to the table "AuthorText". It contains the names and email addresses of all associated authors. These values are generated from the author area, which is normally on the first page of each paper. In the database schema this attribute consists of three values. First, the complete text, which contains the whole author area. Second, the email text, which are all email addresses separated from the complete text. Finally, a list with all authors. The author area is the only area that is not prepossessed.

\myfig{scientific_articles_tree}
      {width=0.90\textwidth}
      {\textbf{Example of a Scientific Article Tree.} In this figure we highlight the hierarchical structure of an typical scientific article. For example it can be composed into multiple sections.}
      {Example of a Scientific Article Tree.}
      {fig:scientific_articles_tree}

One of the most important characteristics are sections, and the underlying structure. \Cref{fig:scientific_articles_tree} shows an example of an article, and the tree-like structure that comes with it. Chapters are non-leaf nodes, and text areas are leaf nodes. This is represented in database schema as two lists, one for subsections and one for text areas. In addition to the lists, each section itself has a section type. This attribute describes whether the section is a section, subsection, or subsubsection. The IMRaD structure information is stored as the IMRaD-Type. As described in \Cref{subsec:generation}, each section may have several IMRaD-Types. Each type of section holds its own list of these types. Hence, subsections and subsubsections keep the same IMRaD-Types as their parent section.

We store word histograms for articles as well as the sections, so we do not have to scan the entire text for each search request. These histograms contain the term frequencies of the corresponding area. Therefore, subsubsections contain the frequencies of their text areas, subsections contain the frequencies of their text areas, and their subsubsections text areas, and so on. Finally, an article holds the term frequencies of the whole document.

The last two attributes of an article are the reference-, and the cited-by-lists. The two lists are used to generate connections between articles. A reference holds the identifier of a referred paper. In turn, the referred paper has an entry with the paper id in the cited-by-list. Additionally, we store the text of whole reference, the authors, and other available information such as publisher, pages, or the volume.

One characteristic over all tables of the schema is, that all extracted text values are available as raw-, and processed text values. Raw represents the text as it was separated from the used framework. Processed is the raw text after the stopword removal and the stemming.

\subsection{Citation Network}
\label{sec:citation_network}

\myfig{dataset_generall}
      {width=0.50\textwidth}
      {\textbf{General Structure of a Citation Network.} The timeline indicates that new articles citing existing articles, and thus there can not be cyclic dependencies between them.}
      {General Structure of a Citation Network}
      {fig:structure_citation_network}

A Citation network represents the relationship between scientific articles. In general citation means that one article mentions the work of another article. Therefore a
reference with the title, the authors and the publication journal is added. \Cref{fig:structure_citation_network} shows the structure of such a network. Scientific articles are nodes, and citations are directed edges between these nodes. The timeline indicates that new articles are citing already existing articles, and thus there can not be cyclic dependencies.

M. Kas ~\cite{kas2011} defined the basic properties of citation networks in their work. In our case the most important ones are:

\begin{itemize}
  \item Directed.
  \item Acyclic.
  \item All edges point backwards in time
  \item Edges are permanent
  \item The existing part is mostly constant. Only the leading edges changes
\end{itemize}

The network is directed and acyclic because each article has a publication date and can only cite previously published articles. Due to these properties edges can only point back in time. The edges of the network have to be permanent because the references of the existing articles never change. When a new node is added it generates edges to existing articles. This means that all other nodes and edges stays constant.

\begin{table}[!b]
  \centering
  \begin{tabular}{ l c }
    \toprule
    \textbf{Number of Nodes}      & $821$  \\ \midrule
    \textbf{Number of Edges}      & $1,716$ \\ \midrule
    \textbf{Longest Path Length}  & $12$   \\ \midrule
    \textbf{Number of Root Nodes} & $107$  \\
    \bottomrule
  \end{tabular}
  \caption[General Properties about the citation network]{\textbf{General Properties about the citation network.} The citation network represents the relationship between our used scientific articles. The number of nodes indicates the number of articles, and the number of edges citations between those articles. There are no cycles inside the graph, and the longest citation chain consists of $12$ articles. There are $107$ articles which has no outgoing edges. That means that none of their referred articles are part of our dataset.}
  \label{tbl:general_properties_about_the_graph}
\end{table}

The main properties of our citation network are shown in \Cref{tbl:general_properties_about_the_graph}. Scientific articles are the nodes, and citations are the edges between these nodes. This means that our network consists of $821$ articles, and $1,716$ citations between these articles. During the generation of our dataset we found cycles due to preprints. Preprints are versions of scientific articles which are not peer reviewed, and published in a scientific journal. In our case we removed all preprints from the dataset. The longest path length indicates that the longest citation chain of our network consists of $12$ articles. The root nodes are nodes without any outgoing edges. In our network are $107$ root articles which cite no other article. This happens because none of their referred articles are part of our dataset.

\begin{figure}[!t]
  \begin{minipage}{\textwidth}
    \begin{minipage}[b]{0.39\textwidth}
      \centering
      \begin{tabular}{ l c }
        \toprule
        \textbf{Max References}    & $98$     \\ \midrule
        \textbf{Mean References}   & $5.8767$ \\ \midrule
        \textbf{Median References} & $2$      \\
        \bottomrule
    \end{tabular} \\
    \vspace*{1cm}
    (a) Properties
  \end{minipage}
  \begin{minipage}[b]{0.59\textwidth}
    \centering
    \includegraphics[width=1.0\textwidth]{figures/in-degree_distribution} \\
    (b) In-Degree Distribution
    \end{minipage}
  \end{minipage}
  \caption[In-Degree Distribution of the Citation Network]{\textbf{In-Degree Distribution.} The in-degree distribution describes how often articles get referred by other articles. The long tail of the distribution indicates that there are a lot of articles which are cited only a few times, and a few articles which are cited more often. There are some outliers with a higher degree than $20$. This can also be seen by the difference between the mean and the median. The maximum number of references in connection with the zoomed view represents that one single article was cited by $98$ other articles.}
  \label{fig:indegree_distribution}
\end{figure}

\Cref{fig:indegree_distribution} describes the in-degree distribution of our citation network. In general, the in-degree of a node is the number of ingoing edges. The in-degree distribution represents the probability distribution of these nodes over the whole network. Regarding a citation network the in-degree of a node is the number of articles which referred to this article. The long tail of the in-degree distribution indicates that there are a lot of articles which are referred only a few times, and a few articles which are referred more often. There are only some articles with an in-degree higher than $20$. The maximum number of references in connection with the zoomed view represents that one single article was referred by $98$ other articles.

\begin{figure}[!t]
  \begin{minipage}[!t]{\textwidth}
    \begin{minipage}[b]{0.39\textwidth}
      \centering
      \begin{tabular}{ l c }
        \toprule
        \textbf{Max References}    & $13$     \\ \midrule
        \textbf{Mean References}   & $2.4034$ \\ \midrule
        \textbf{Median References} & $2$      \\
        \bottomrule
    \end{tabular} \\
    \vspace*{1cm}
    (a) Properties
  \end{minipage}
  \begin{minipage}[b]{0.59\textwidth}
    \centering
    \includegraphics[width=1.0\textwidth]{figures/out-degree_distribution} \\
    (b) Out-Degree Distribution
    \end{minipage}
  \end{minipage}
\caption[Out-Degree Distribution of the Citation Network]{\textbf{Out-Degree Distribution.} The out-degree distribution describes how often articles refer other articles. The long tail of the distribution indicates that there are a lot of articles which refer less than $7$ other articles, and only few with refer to more articles. Mean and median of the outgoing edges are low, because not every refereed article is part of our dataset. By the small difference between the mean and the median can be seen that there are less outliers. The maximum number of references represent that the highest number of a single article refers other articles is $13$.}
\label{fig:outdegree_distribution}
\end{figure}

The out-degree distribution and their properties are displayed in \Cref{fig:outdegree_distribution}. In contrast to the in-degree distribution, the out-degree distribution describes the number of outgoing edges. Regarding to the citation network, the out-degree of a node can be described as the number of articles which gets referred by this article. The long tail of the out-degree distribution indicates that there are a lot of articles which refer less than $7$ other articles, and only few with refer $7$ or more articles. Mean and median of the outgoing edges are low, because not every refereed article is part of our dataset. By the small difference between mean and median we can also see that there are less outliers. The maximum number of references represent the highest number of a single article refers to other articles, which is in our case $13$.

\section{Model}
\label{sec:model}

Write general about the implemented information retrieval model ...

\subsection{Query Structure}

Write how a query looks like - explicit and implicit

\subsection{Introducing IMRaD Structure Features into Weighing Schemes}

Write how the algorithms described in \Cref{sec:unstructured_text_Retrieval} are modified for IMRaD structure features.

\chapter{Evaluation Results}
\label{cha:evaluation-results}
Mean Average Precision is used to compare the search algorithms.


\section{Leveraging IMRaD Structure Features}
Does it make sense to use aditional information about the chapter in the search step?
\subsection{Explicit Search using N-Grams}

\begin{table}
  \begin{adjustwidth}{-2cm}{}
    \begin{tabular}{ | c | C{1.5cm} | C{2.1cm} || C{2.1cm} | C{2.1cm} | C{2.1cm} | C{2.1cm} | C{2.1cm} |}
      \hline
      \rowcolor{lightblue}
      \textbf{n} & \textbf{\# queries} & \textbf{Using IMRaD Chapter Features} & \textbf{Term Frequency} & \textbf{Tf-idf} & \textbf{Ranked Boolean Retrieval} & \textbf{Okapi BM25} & \textbf{Divergence from Randomness} \\ \hline
      \multirow{2}{*}{2} & \multirow{2}{*}{8770} & No  & 0.0882 & 0.1128 & 0.1035 & 0.0442 & 0.0  \\ \cline{3-8}
                                                && Yes & 0.0696 & 0.0897 & 0.0638 & 0.045  & 0.0  \\ \hline \hline
      \multirow{2}{*}{3} & \multirow{2}{*}{7070} & No  & 0.1038 & 0.1382 & 0.1198 & 0.064  & 0.0  \\ \cline{3-8}
                                                && Yes & 0.0785 & 0.1042 & 0.0713 & 0.0628 & 0.0  \\ \hline \hline
      \multirow{2}{*}{4} & \multirow{2}{*}{5589} & No  & 0.1197 & 0.1547 & 0.1336 & 0.0739 & 0.0  \\ \cline{3-8}
                                                && Yes & 0.0894 & 0.1167 & 0.0787 & 0.0734 & 0.0  \\ \hline
    \end{tabular}
  \caption[Ranking results with explicit search]{Ranking results of the used weigthing schemes using explicit search}
  \label{tbl:ranking_result_explicit}
  \end{adjustwidth}
\end{table}

TODO: Okapi - Better param search, increase n

\subsection{Implicit Search using Scientific Articles}

... more like this.

\begin{table}
  \begin{adjustwidth}{-2cm}{}
    \begin{tabular}{ | C{2.1cm} | C{2.1cm} | C{2.1cm} | C{2.1cm} | C{2.1cm} | C{2.1cm} | C{2.1cm} | }
      \hline
      \rowcolor{lightblue}
      \textbf{\# Scientific Articles} & \textbf{Using IMRaD Chapter Features} & \textbf{Term Frequency} & \textbf{Tf-idf} & \textbf{Ranked Boolean Retrieval} & \textbf{Okapi BM25} & \textbf{Divergence from Randomness} \\ \hline
      \multirow{2}{*}{714} & No  & 0.1186 & 0.1163 & 0.0466 & 0.0554 & 0.0 \\ \cline{2-7}
                           & Yes & 0.1463 & 0.1613 & 0.0506 & 0.0882 & 0.0 \\ \hline
    \end{tabular}
  \caption[Ranking results using scientific articles]{Ranking results of the used weigthing schemes using scientific articles}
  \label{tbl:ranking_result_full}
  \end{adjustwidth}
\end{table}


\section{Chapter Based Search}
Does the search result improve if only a single chapter is used for searching? \\
Which chapter has the most influence?

Number of scientific articles: 714

\begin{table}
  \centering
  \begin{tabular}{ | c | c | c | c | c | c | }
    \hline
     & \textbf{Introduction} & \textbf{Background} & \textbf{Methods} & \textbf{Results} & \textbf{Discussion} \\ \hline
    \textbf{Introduction} & 0.1242 & 0.1226 & 0.1095 & 0.1092 & 0.1049 \\ \hline
    \textbf{Background}   & 0.1454 & 0.1249 & 0.1331 & 0.1255 & 0.1106 \\ \hline
    \textbf{Methods}      & 0.0947 & 0.0857 & 0.1017 & 0.0897 & 0.0668 \\ \hline
    \textbf{Results}      & 0.0877 & 0.0783 & 0.0815 & 0.0783 & 0.0631 \\ \hline
    \textbf{Discussion}   & 0.1188 & 0.1078 & 0.0957 & 0.0914 & 0.084  \\ \hline
  \end{tabular}
  \caption[Chapter based Serch using Tf-idf]{Chapter based Serch using Tf-idf}
  \label{tbl:ranking_result_full}
\end{table}

\chapter{Conclusion}
\label{cha:conclusion}

%In the final chapter we 

\section{Summary}
\label{sec:summary}

We started our work with an motivation section, where we discuss the fast grow of the internet and the resulting importance of search engines. Therefore, search engines help to reduce the time required to find a piece of information, and minimize the number of information sources that need to be searched. In the field of science they help to simplify literature search. 

An advantage of scientific articles is that they have a common structure to increase the readability. This structure is known is IMRaD (Introduction, Method, Results and Discussion). In our work we tackle the question whether it is possible to improve the search result quality while searching for scientific works by using IMRaD structure information. Specifically, 
\begin{enumerate}[label=(\alph*)]
  \item Does the search result improve for explicit search using queries?
  \item Does the search result improve for implicit search using complete scientific papers?
  \item Does the search result improve if only a single chapter of the scientific paper is used for searching?
\end{enumerate}
In the related work section we describe the definition of an information retrieval model. Afterwards, we discuss the $3$ classical models for unstructured text retrieval. First, in the boolean model, documents and queries are represented as sets. Terms are stitched together with boolean operators to formulate user queries. Second, in the vector model, documents and queries are represented asa vector in a t-dimensional space. Third, in the probabilistic model, documents and queries are represented based on probability theory. Specifically by estimating the probability of a term appearing in a relevant document.

Additionally, we describe extensions of the vector-, and the probabilistic model. First, the TF-IDF model is based on the vector space model, and one of the most popular weighting schemes in information retrieval. Second, the BM$25$ model is based on the probabilistic model. It is the result of several experiments by Robertson et. al ~\cite{RobertsonWHGL92, RobertsonWJHG93, RobertsonWJHG94}. Third, the Divergence from Randomness model was introduced by Amati and Rijsbergen ~\cite{AmatiR02} and is a probabilistic model that exhibits characteristics of a language model as well.

Next, we discuss techniques of structured text retrieval models. There we focus specially on $5$ ranking strategies known as contextualization, propagation, aggregation, merging, and zone scores. The model based on zone scores is proposed by Manning et al. ~\cite{manning2008}, and also known as Ranked Boolean Retrieval.

In the last part of the related work we focus on the IMRaD structure in scientific articles. Sollaci and Pereira ~\cite{Sollaci-The-2004} describe in their work that the IMRaD structure began to be adopted in the 1940s, and became the standard format for scientific articles in the 1970s. Furthermore, we discuss IMRaD structure distributions as proposed by Bertin et al. ~\cite{bertin2013}, and how IMRaD structure can be leveraged in information retrieval systems.

In the methods section we started with the description of our generated dataset. There we have $821$ scientific articles in our dataset, and we added additional information like IMRaD mappings, and links between the articles based on citations. Furthermore, we defined our database schema with respect to the article structure and analyze the citation network.

Afterwards, we describe our introduced system and the underlying model. There we design our system to compare various common ranking algorithms. Hence, the ranking algorithms requires to be changeable easily. In addition, our model was designed to work with unstructured as well as structured data. This is reflected by the query language.

In the results and discussion section we describe a measurement for the performance of ranking algorithms. This measurement is required to compare our proposed ranking algorithms.

Next, we list our study results, and discuss a first interpretation. 

% research questions iterrieren


% new chapter
% summarize discussion of single results
% discuss overall results

\section{Future Work}
\label{sec:future_work}

% dataset increase + assumptions check (maybe other datasets like PLOS)

% bm25 better param search

% idf... imrad based bzw.. not wenn umgedreht

% performance improve (db, serach with document..)

% other dechnoligies of structured text retrieval (not mean over all imrad types)


%\appendix                       %% closes main document, appendix follows until end; only available in book-classes
%\addpart*{Appendix}             %% adding Appendix to tableofcontents

\printbibliography              %% remove, if using BibTeX instead of biblatex
% \include{further_ressources}  %% this is a suggestion: you have to create this file on demand






%%%% end{document}
\end{document}
%% vim:foldmethod=expr
%% vim:fde=getline(v\:lnum)=~'^%%%%\ .\\+'?'>1'\:'='
%%% Local Variables:
%%% mode: latex
%%% mode: auto-fill
%%% mode: flyspell
%%% eval: (ispell-change-dictionary "en_US")
%%% TeX-master: "main"
%%% End:
