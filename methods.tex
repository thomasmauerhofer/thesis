\chapter{Materials and Method}
\label{cha:materials_and_method}

\section{Dataset}
\label{sec:dataset}

\subsection{Creation}
\label{sec:creation}

We created our used dataset from approximately 3000 scientific articles in pdf format. An important point was that these articles contained various information, and thereby we ensured that they come from different scientific fields. To be able to use the data we had to separate them from the pdfs and add additional information first.

We used a text mining pre-processing technique as introduced by Vijayarani et al.~\cite{Vijayarani2015} to separate the data from the pdf. This technique consists of three key steps, which are called the extraction-, the stopwords removal-, and the stemming-step. At first we seperated the article structure, and the raw text from the pdf. For those we used a framework descriped in \Cref{sec:extract_document_structure}. The stopwords removal-, and the stemming step was done as descriped in \cref{sec:stop_words} and \Cref{sec:stemming} (TODO: more detailed).

TODO: Write about the imrad type of an chapter

At last we performed a semi-automated annotation to link the references of the scientific articles. Thereby similarities about the references of an article and the titles of all other articles are compared as described in \Cref{sec:text_similarities}, and if they exceed a given threshold a recommentation to create a link between those two was given.

Each data record was created in such a way that it can be transferred directly to the database schema showen in \Cref{fig:database_schema}.

\myfig{database_schema}
      {width=0.70\textwidth}
      {Database Schema for the used Dataset}
      {Database Schema for the used Dataset}
      {fig:database_schema}

\subsection{Structure of a Scientific Article}
\label{sec:structure_scientific_article}

When looking at the scientific article in the database schema in \Cref{fig:database_schema} there are various member attributes. The first important point is that all text values are available raw and processed. The term raw stands for the text being saved as it was separated. Processed text is the raw text after the stopword removal-, and the stemming step.

The author text attribute contains the names and e-mail adresses of all authors. Noch das der complette text, und eine extra liste mit autoren gespeichert werden...

section - member... imrad - section type, and subsections - text list und subsections list

\myfig{scientific_articles_tree}
      {width=0.90\textwidth}
      {Scientific Writing represended as a Tree}
      {Scientific Writing represended as a Tree}
      {fig:scientific_articles_tree}

\Cref{fig:scientific_articles_tree} showes the scientific article with regard to the individual sections. It is easy to see that the article itself is the root node, chapters are non-leaf nodes, and text areas are the leaf nodes.

word hist

references, cited by

\subsection{Citation Network}
\label{sec:citation_network}

\myfig{dataset_generall}
      {width=0.50\textwidth}
      {General Structure of a Paper Network}
      {General Structure of a Paper Network}
      {fig:dataset_generall}

\begin{table}
  \centering
  \begin{tabular}{ l c }
    \toprule
    \textbf{Number of Nodes} & 821 \\ \midrule
    \textbf{Number of Edges} & 1716 \\ \midrule
    \textbf{Number of Components} & 1 \\ \midrule
    \textbf{Number of Cycles} & 0 \\ \midrule
    \textbf{Longest Path} & 12 \\ \midrule
    \textbf{Number of Root Nodes} & 107 \\
    \bottomrule
  \end{tabular}
  \caption[General Properties about the graph in the used dataset]{General Properties about the graph}
  \label{tbl:general_properties_about_the_graph}
\end{table}


\myfig{in-degree_distribution}
      {width=0.80\textwidth}
      {In-Degree Distribution of the used Dataset}
      {In-Degree Distribution of the used Dataset}
      {fig:in-degree_distribution}

\begin{table}
  \centering
  \begin{tabular}{ l c }
    \toprule
    \textbf{Max Degree} & 98 \\ \midrule
    \textbf{Mean Degree} & 5.8767 \\ \midrule
    \textbf{Median Degree} & 2 \\
    \bottomrule
  \end{tabular}
  \caption[Properties of the ingoing edges in the used dataset]{Properties of the ingoing edges}
  \label{tbl:properties_ingoing_edges}
\end{table}

\myfig{out-degree_distribution}
      {width=0.80\textwidth}
      {Out-Degree Distribution of the used Dataset}
      {Out-Degree Distribution of the used Dataset}
      {fig:out-degree_distribution}

\begin{table}
  \centering
  \begin{tabular}{ l c }
    \toprule
    \textbf{Max Degree} & 13 \\ \midrule
    \textbf{Mean Degree} & 2.4034 \\ \midrule
    \textbf{Median Degree} & 2 \\
    \bottomrule
  \end{tabular}
  \caption[Properties of the outgoing edges in the used dataset]{Properties of the outgoing edges}
  \label{tbl:properties_outgoing_edges}
\end{table}

\subsubsection{Data cleaning}
\label{subsubsec:data_cleaning}
Write something about the cycles in the network. preprints cite each other

\myfig{preprint_problem}
      {width=0.30\textwidth}
      {Cyclic dependency in a directed graph}
      {Cyclic dependency in a directed graph}
      {fig:preprint_problem}

\section{Model}
\label{sec:model}

Write general about information retrieval models as descriped in ~\cite{ModernInvormationRetrieval1999} page 57

\subsection{Query Structure}

Write how a query looks like - explizit and implizit

\subsection{Introducing IMRaD Structure Features into Weighing Schemes}

Write how the algorithms described in \cref{sec:ranking_algorithms} are modified for imrad structure features.
