\chapter{Materials and Method}
\label{cha:materials_and_method}

\section{Dataset}
\label{sec:dataset}

\subsection{Generation}
\label{subsec:generation}

We created our used dataset from approximately 3000 scientific articles in PDF format. An important point was that these articles contained various information, and thereby we ensured that they come from different scientific fields. To be able to use the data we had to separate them from the PDFs and add additional information first.

We used a text mining pre-processing technique as introduced by Vijayarani et al.~\cite{Vijayarani2015} to separate the data from the PDF. This technique consists of three key steps, which are called the extraction-, the stopwords removal-, and the stemming-step. At first we seperated the article structure, and the raw text from the PDF. For those we used a framework descriped in \Cref{subsec:extract_document_structure}. The stopwords removal-, and the stemming step was done as descriped in \cref{subsec:stop_words} and \Cref{subsec:stemming}.

\begin{table}[!b]
  \centering
  \begin{tabular}{C{2.6cm} C{2.7cm} C{1.5cm} C{1.5cm} C{1.5cm} C{1.5cm}}
    \toprule
    & \multicolumn{3}{c}{\textbf{per Section Title}} & \multicolumn{2}{c}{\textbf{Overall}} \\
    \textbf{IMRaD Type} & \textbf{Section Title} & \textbf{\# Paper} & \textbf{Percent} & \textbf{\# Paper} & \textbf{Percent} \\ \midrule
    Introduction & Introduction & $822$ & $100\%$ & $822$ & $100\%$ \\ \midrule
    Related Work & Related Work & $465$ & $56.57\%$ & $465$ & $56.57\%$ \\ \midrule
    \multirow{3}{*}{Methods} & Method & $97$ & $11.8\%$ & \multirow{3}{*}{$312$} & \multirow{3}{*}{$37.96\%$} \\
    & Model & 134 & $16.3\%$ \\
    & Approach & 81 & $9.85\%$ \\ \midrule
    \multirow{3}{*}{Result} & Experience & $396$ & $48.18\%$ & \multirow{3}{*}{$687$} & \multirow{3}{*}{$83.58\%$} \\
    & Result & 163 & $19.83\%$ \\
    & Evaluation & 128 & $15.57\%$ \\ \midrule
    \multirow{3}{*}{Discussion} & Conclusion & $581$ & $70.68\%$ & \multirow{3}{*}{$773$} & \multirow{3}{*}{$94.04\%$} \\
    & Discussion & 179 & $21.78\%$ \\
    & Future Work & 13 & $1.58\%$ \\
    \bottomrule
  \end{tabular}
  \caption[Mapping of the Section Titles to IMRAD-Types]{\textbf{Mapping of the Section Titles to IMRAD-Types.} This Table shows which section titles was used to generate the IMRaD structure information. Additionally the relation between those titles, and how often they occures in the used dataset is displayed.}
  \label{tbl:mapping_section_names}
\end{table}

One of the additional details we had to add was IMRaD structure information. The IMRaD structure provides related work to be part of the introduction. Since most of our papers had an own section titled related work we chose to add an additional type for it. We were able to classifify the IMRaD structure with simple keyword detection in the section titles. \Cref{tbl:mapping_section_names} shows the mapping between IMRAD-Types and these titles. Because it was not really possible to identify method sections by using the section titles we used information about their position in the article. We manage that by using the related work as upper bound, and the result as lower bound. We classifying all sections between these two bounds as a method section. If the related work section was not available we use the introduction section as upper bound, or if the result section was not available we use the discussion section as lower bound. If one of the two bounds could not be set, we discarded the scientific article. Chapters can have several IMRaD-Types. For example if a section was titled \textit{"Results and Discussion"} it belongs to the types result and discussion.

Another additional details we had to add was links between the scientific articles. For this we performed a semi-automated annotation. Thereby similarities about the references of an article and the titles of all other articles are compared as described in \Cref{sec:text_similarities}, and if they exceed a given threshold a recommentation to create a link between those two was given.

We created each data record in such a way that it can be transferred directly to the database schema showen in \Cref{fig:database_schema}. To reduce noise during the evaluation we removed all articles without any connection to other articles. Finally we had 821 scientific articles in our dataset.

\myfig{database_schema}
      {width=0.70\textwidth}
      {Database Schema for the used Dataset}
      {Database Schema for the used Dataset}
      {fig:database_schema}

\subsection{Structure of a Scientific Article}
\label{sec:structure_scientific_article}

When looking at the scientific article in the database schema in \Cref{fig:database_schema} there are various member attributes. The first important point is that all text values are available raw and processed. The term raw stands for the text being saved as it was separated. Processed text is the raw text after the stopword removal-, and the stemming step.

The author text attribute contains the names and email adresses of all authors. Thereby this attribute has three values. The complete text, the text reduced to emails, and a list of all authors. This part of the article is the only one that is not preprocessed.

\myfig{scientific_articles_tree}
      {width=0.90\textwidth}
      {\textbf{Scientific Article Tree.} In this figure we highlight the hiracical structure of an typical scientific article. For example it can be composed into multiple sections.}
      {Scientific Article represended as a Tree}
      {fig:scientific_articles_tree}

One of the most important characteristics are the sections, and the underlying structure that comes with it. \Cref{fig:scientific_articles_tree} showes that scientific articles are structured like a tree. It is easy to see that the chapters are non-leaf nodes, and text areas are the leaf nodes. This is reflected in database schema with two lists. One for the subsections, and one for text areas. In addition to the lists, each section itself has a section type. This attribute reflects whether the section is a section, subsection, or subsubsection. The IMRaD structure information is stored as the IMRaD-Type. As described in \Cref{subsec:generation}, each section can have several IMRaD-Types. Each type of section holds its own list of these types. This means that subsections and subsubsections keeps the same IMRaD-Types as the parent section.

We stored word histograms for the article, and the sections so we do not have to discover the complete text for each search request. These histograms contain the term frequencies of the underlying area. So the subsubsections contain the frequencies of their text areas, the subsections contain the frequencies of their text areas, and their subsubsections text areas, and so on. Finally the article holds all term frequencies of the whole document.

The last two attributes of the article are the reference-, and the cited-by-list. If an id is set in the reference, the referenced paper has an entry in the cited-by-list. Additional we stored the text of whole reference, the authors, and other available information like publisher, pages, volume, etc.

\subsection{Citation Network}
\label{sec:citation_network}

\myfig{dataset_generall}
      {width=0.50\textwidth}
      {General Structure of a Citation Network}
      {General Structure of a Citation Network}
      {fig:structure_citation_network}

A Citation networks represents the relationship between scientific articles. \Cref{fig:structure_citation_network} shows the structure, and the properties ~\cite{kas2011} of such a network. Therby the articles regards as nodes, and an edge from node i to node j is added if i cites j. The network involves ofer time. ie. new articles cites existing onses. The timeline indicates that new articles citing existing articles, and thus there can not be cyclic dependencies.

\begin{table}[!b]
  \centering
  \begin{tabular}{ l c }
    \toprule
    \textbf{Number of Nodes}      & $821$  \\ \midrule
    \textbf{Number of Edges}      & $1,716$ \\ \midrule
    \textbf{Number of Cycles}     & $0$    \\ \midrule
    \textbf{Longest Path Length}  & $12$   \\ \midrule
    \textbf{Number of Root Nodes} & $107$  \\
    \bottomrule
  \end{tabular}
  \caption[General Properties about the citation network]{General Properties about the citation network}
  \label{tbl:general_properties_about_the_graph}
\end{table}

Warum keine cyles, und der connection der zeitlichen abhengigkeit

eine exception sind preprints - (paper before sie peer reviewed werden) - hatte 5 drinnen, removed...

longest path - was heißt es in context of citation graph, es gibt einen ausblick über die zeitspanne da nacheinander. Diese paper sind related...

The main properties of our citation network are shown at \Cref{tbl:general_properties_about_the_graph}(TODO: describe importat points).
Write something about the cycles in the network. preprints cite each other, and why they are removed for the network.
%Three important points about the network
%One important point is that there are no cycles, which can be also seen by the number of strongly connected components. The longest citation chain over the timeline is defined by the longest path.%
\begin{minipage}[!t]{\textwidth}
  \begin{minipage}[b]{0.39\textwidth}
    \centering
    \begin{tabular}{ l c }
      \toprule
      \textbf{Max Degree}    & $98$     \\ \midrule
      \textbf{Mean Degree}   & $5.8767$ \\ \midrule
      \textbf{Median Degree} & $2$      \\
      \bottomrule
  \end{tabular} \\
  \vspace*{1cm}
  (a) Properties
\end{minipage}
\begin{minipage}[b]{0.59\textwidth}
  \centering
  \includegraphics[width=1.0\textwidth]{figures/in-degree_distribution} \\
  (b) In-Degree Distribution
  \end{minipage}
  \captionof{figure}[In-Degree Distribution of the Citation Network]{\textbf{In-Degree Distribution} beschreiben indegree(wie oft hat ein paper ein anderes zitiert) mean wert ist 2, also nur 2 mal zitiert. gibt aber ausreiser, da einige sehr relevant sind.}
  \label{fig:indegree_distribution}
\end{minipage}

The in-degree distribution and their properties How is it represented (citations to papers)

\begin{minipage}[!t]{\textwidth}
  \begin{minipage}[b]{0.39\textwidth}
    \centering
    \begin{tabular}{ l c }
      \toprule
      \textbf{Max Degree}    & $13$     \\ \midrule
      \textbf{Mean Degree}   & $2.4034$ \\ \midrule
      \textbf{Median Degree} & $2$      \\
      \bottomrule
  \end{tabular} \\
  \vspace*{1cm}
  (a) Properties
\end{minipage}
\begin{minipage}[b]{0.59\textwidth}
  \centering
  \includegraphics[width=1.0\textwidth]{figures/out-degree_distribution} \\
  (b) Out-Degree Distribution
  \end{minipage}
  \captionof{figure}[Out-Degree Distribution of the Citation Network]{\textbf{Out-Degree Distribution} beschreiben outdegree etc...}
  \label{fig:indegree_distribution}
\end{minipage}

The out-degree distribution and their properties How is it represented (citations to papers)

\section{Model}
\label{sec:model}

Write general about information retrieval models as descriped in ~\cite{ModernInvormationRetrieval1999} page 57

\subsection{Query Structure}

Write how a query looks like - explizit and implizit

\subsection{Introducing IMRaD Structure Features into Weighing Schemes}

Write how the algorithms described in \cref{sec:ranking_algorithms} are modified for imrad structure features.
