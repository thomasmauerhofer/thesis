\chapter{Method}
\label{cha:method}

\section{Dataset}
\label{sec:dataset}

\subsection{Creation}
The used dataset was created from approximately 3000 scientific articles in pdf format. An important point was that those articles contained various information, and thereby it was ensured that they come from different scientific fields.

A text mining pre-processing technique as introduced by Vijayarani et al.~\cite{Vijayarani2015} was used to separate the data from the pdf, and store it into a database.
This technique consists of three key steps, which are called the extraction-, the stopwords removal-, and the stemming-step. At first the article structure, and the raw text was seperated from the pdf. For those a framework descriped in \cref{sec:extract_document_structure} was used. The stopwords removal-, and the stemming step was done as descriped in \cref{sec:stop_words} and \cref{sec:stemming} (TODO: more detailed).

TODO: Write about the imrad type of an chapter

To link the references of the scientific articles with others, and vice versa, a semi-automated annotation was performed. Thereby similarities about the references of an article and the titles of all other articles are compared as described in \cref{sec:text_similarities}, and if they exceed a given threshold a recommentation to create a link between those two was given.

Each data record was created in such a way that it can be transferred directly to the database schema showen in \cref{fig:database_schema}.

\myfig{database_schema}
      {width=0.70\textwidth}
      {Database Schema for the used Dataset}
      {Database Schema for the used Dataset}
      {fig:database_schema}

\subsection{Structure}
\label{sec:structure}

Write about the generall structure

\myfig{scientific_articles_tree}
      {width=0.90\textwidth}
      {Scientific Writing represended as a Tree}
      {Scientific Writing represended as a Tree}
      {fig:scientific_articles_tree}

\Cref{fig:scientific_articles_tree} showes the scientific article with regard to the individual sections. It is easy to see that the article itself is the root node, chapters are non-leaf nodes, and text areas are the leaf nodes.

\subsection{Scientific Articles as a Network}
\label{sec:scientific_articles_as_a_network}

\myfig{dataset_generall}
      {width=0.50\textwidth}
      {General Structure of a Paper Network}
      {General Structure of a Paper Network}
      {fig:dataset_generall}

\begin{table}
  \centering
  \begin{tabular}{ | c | c | }
    \hline
    \textbf{Number of Nodes} & 821 \\ \hline
    \textbf{Number of Edges} & 1716 \\ \hline
    \textbf{Number of Components} & 1 \\ \hline
    \textbf{Number of Cycles} & 0 \\ \hline
    \textbf{Longest Path} & 12 \\ \hline
    \textbf{Number of Root Nodes} & 107 \\ \hline
  \end{tabular}
  \caption[General Properties about the graph in the used dataset]{General Properties about the graph}
  \label{tbl:general_properties_about_the_graph}
\end{table}


\myfig{in-degree_distribution}
      {width=0.80\textwidth}
      {In-Degree Distribution of the used Dataset}
      {In-Degree Distribution of the used Dataset}
      {fig:in-degree_distribution}

\begin{table}
  \centering
  \begin{tabular}{ | c | c | }
    \hline
    \textbf{Max Degree} & 98 \\ \hline
    \textbf{Mean Degree} & 5.8767 \\ \hline
    \textbf{Median Degree} & 2 \\ \hline
  \end{tabular}
  \caption[Properties of the ingoing edges in the used dataset]{Properties of the ingoing edges}
  \label{tbl:properties_ingoing_edges}
\end{table}

\myfig{out-degree_distribution}
      {width=0.80\textwidth}
      {Out-Degree Distribution of the used Dataset}
      {Out-Degree Distribution of the used Dataset}
      {fig:out-degree_distribution}

\begin{table}
  \centering
  \begin{tabular}{ | c | c | }
    \hline
    \textbf{Max Degree} & 13 \\ \hline
    \textbf{Mean Degree} & 2.4034 \\ \hline
    \textbf{Median Degree} & 2 \\ \hline
  \end{tabular}
  \caption[Properties of the outgoing edges in the used dataset]{Properties of the outgoing edges}
  \label{tbl:properties_outgoing_edges}
\end{table}

\subsection{Data cleaning}
\label{subsec:data_cleaning}
Write something about the cycles in the network. preprints cite each other

\myfig{preprint_problem}
      {width=0.30\textwidth}
      {Cyclic dependency in a directed graph}
      {Cyclic dependency in a directed graph}
      {fig:preprint_problem}


\section{Preprocessing}
\label{sec:preprocessing}

Write how the underlying framework was used, stemming, remove stopwords, classify chapters...

\section{Introducing IMRaD Structure Features into Weighing Schemes}

Write how the algorithms described in \cref{sec:ranking_algorithms} are modified for imrad structure features.
