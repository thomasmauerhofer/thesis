\documentclass[a4paper, 12pt]{scrartcl}

\usepackage{ngerman}
\usepackage[utf8]{inputenc}
\usepackage[T1]{fontenc}
\usepackage{ae,aecompl}
\usepackage{psfrag}
\usepackage[table]{xcolor}
\usepackage[automark]{scrpage2}
\usepackage[pdftex]{graphicx}
\pdfcompresslevel=9
\usepackage[pdftex=true,
    backref,
    pagebackref=false,
    colorlinks=true,
    bookmarks=true,
    bookmarksopen=false,
    bookmarksnumbered=false,
    pdfpagemode=None
  ]{hyperref}
\DeclareGraphicsExtensions{.pdf}

\hypersetup{
  pdftitle={}, %%
  pdfauthor={}, %%
  pdfsubject={}, %%
  pdfcreator={Accomplished with LaTeX2e and pdfLaTeX with hyperref-package.}, %% 
  pdfproducer={}, %%
  pdfkeywords={} %%
}

\title{The use of structured text retrieval to search for scientific work}

\author{Betreuer: \\
Dipl.-Ing. Dr.techn. Roman Kern \\ \\ \\ \\ \\
Thomas Mauerhofer (1031957)}

\date{Graz, am \today{}}

\bibliographystyle{plain}

\pagestyle{scrheadings}

\definecolor{lightblue}{rgb}{0.93,0.95,1.0}

\begin{document}

 \maketitle
 
\newpage

\tableofcontents

\newpage

\section{Introduction}
\label{sec:Introduction}

In der Gegenwart gehört die Suche nach Informationen via Internet und im Besonderen Google, zum Alltag der meisten Menschen. Hatte Google im Jahr 1998 noch 10.000 Suchanfragen am Tag, wurde dieselbe Zahl 
an Anfragen 2006 in einer Sekunde getätigt. (Quelle: \url{http://www.internetlivestats.com/google-search-statistics/} --> Zugriff am 15.05.2017). Dadurch ist es nur naheliegend, dass auch die Anzahl an zugängichen
Informationen stetig zunimmt. Täglich werden z.B. unzählige neuen Webseiten erstellt, Artikel geschrieben und wissenschaftliche Arbeiten veröffentlicht. Um
diese Menge an Informationen zu managen und zwischen relevanten und nicht relevanten Quellen zu unterscheiden, verwendet man unterschiedliche Suchmaschinen. 
Dabei soll die Suche simpel aber dennoch genau sein.

Betrachtet man Suchmaschinen für wissenschaftliche Arbeit und Forschung existiert eine große Auswahl an Möglichkeiten. Eine der bekanntesten Suchmaschinen im wissenschaftlichen Bereich ist wohl Google
Scholar, welches eine einfache Eingabe zur Suche verwendet und die gefundenen Arbeiten entsprechend ihrer Relevanz listet. Diese Listungen enthalten verschiedene Dateiformate aus unterschiedlichen Jahren zu 
diversen Themen, die nicht immer die Kriterien der Suchanfrage wiedergeben. Gerade für wissenschaftliche Arbeiten und Forschung ist es allerdings wichtig, ein präzises Suchergebnis zu erhalten. 

Diese Arbeit befasst sich mit der Verbesserung von Suchanfragen und deren Ergebnissen für wissenschaftliche Arbeiten in PDF Format. Recherchiert man z.B. einen Autor, erhält man mit den gängigen Suchmaschinen
nicht nur die veröffentlichten Artikel und Bücher, sondern häufig auch alle Quellen in welchen dieser zitiert wurde. Ziel dieser Arbeit ist es die Eingabe des Suchbegriffes robuster zu gestalten, die 
Usability der Suchmaschine zu erhöhen und so treffsicherere Ergebnisse zu liefern. Dies geschieht im Hilfe von einfachen search queries Strukuren, durch selbsterklärendes Front End, Spelling Checks und 
der Bearbeitung und Beurteilung im Back End. 

\section{Related Work}

Wie bereits in der \nameref{sec:Introduction} erwähnt befasst sich die geplante Masterarbeit mit der Verbesserung von Suchanfragen und deren Ergebnissen für wissenschaftliche Arbeiten in PDF Format. Laut 
\cite{KGJK14} sind alle wissenschaftliche Arbeiten in ähnlichen Strukturen aufgebaut. Diese Strukturen unterteilen sich in capters, sections, subsections and so on, welche sich gut in dem \cite{RNBY99} 
beschriebene Structured Text Retrieval Model anwenden lassen. Dieses Model beschreibt den Umgang von Suchanfragen und deren Ergebnissen für wissenschaftliche Arbeiten, sowie deren Bearbeitung und Beurteilung 
im Backend durch die Verwendung der Metainformation über die Struktur des Dokuments. Um die Dokumente nach ihrer Priorität zu listen kommen verschiedene Ranking strategies z.B. Jelinek-Mercer smoothing, zum 
Einsatz. Durch Erweiterungen wie contextualization or aggregation strategies werden die Ergebnisse des rankings noch verbessert.

Um einfache und robuste search queries Strukuren zu gewährleisten wird die Syntax wie in \cite{Coh03} gestaltet. Dabei bestehen die queries aus mehreren Termen, welche in der Form \textit{label}:\textit{keyword} 
gegliedert sind. Zusätzlich können Termen ein \textit{+} vorangestellt werden, um sie besonders hoch zu priorisieren. Durch diese Struktur wird das user interface dann mit einer erweiteren Suche ausgebaut, um 
die usability stark zu erhöhen und ein selbsterklärendes Front End zu schaffen.

Der letzte Punkt um die usability zu verbessern sind spelling checks. Diese können wie in \cite{SPCB13} beschrieben implementiert werden. Hierbei wird mithilfe der Levenshtein Distance ermittelt, ob ein Wort
korrekt geschrieben wurde. Falls dies nicht der Fall ist werden ähnliche Wörter gesucht und vorgeschlagen. 

\section{Aufbau}

Zu Beginn der Arbeit wird eine Grundstruktur für das Projekt geschaffen. Durch das microframework Flask (\url{http://flask.pocoo.org/}) werden das Frontend in Javascript, und das Backend in Python geschrieben.
Durch einfache requests lassen sich Eingaben von der Userseite zur Serverseite schicken. Dort wird die Anfrage dann bearbeitet und ein response versendet. Dies bildet ein solides Grundgerüst für das 
structured text retrieval model.

Python besitzt eine eigene library um Verbindungen zu MySQL Datenbanken herzustellen. Letzere wird verwendet um eine Datenbank wie in \cite{YA94} beschrieben, zu generieren. Die Befüllung mit 
wissenschaftlichen Arbeiten in der Datenbank erfolgt mit den in \cite{KGJK14} beschriebenen Tool.

Daran anschließend kann ein ranking system nach \cite{MRS08, RNBY99} implementiert werden. 

Als nächster Schritt folgt die Anwendung der spelling check auf die search queries um die Usability zu erhöhen.

Zur Verbesserung des Userinterface, wird zuletzt eine erweiterte Suche angewendet. Dadurch muss man den Syntax der search queries nicht mehr kennen, was ein selbsterklärendes Userinterface zur Folge hat. 

\section{Gliederung}

\begin{enumerate}
 \item Introduction
 \item Related Work
 \begin{enumerate}
  \item Unsupervised document structure analysis
  \item Structured Text Retrieval
  \item Structured-Text Retrieval System with an Object-Oriented Database System
  \item Search queries
  \item Real-word error detection and correction
 \end{enumerate}
 \item Implementierung
 \begin{enumerate}
  \item Basisstruktur
  \item Aufbau der Datenbank
  \item Userinterface
  \item Satisfaction of searchqueries
  \item Ranking system
  \item Spelling checks
 \end{enumerate}
 \item Result
 \item Conclusion
\end{enumerate}

\section{Auswahlbibliografie}

\bibliography{literature}
\nocite{*}

\section{Zeitplan}


\begin{tabular}{ | l | l | c | }
  \hline
  \rowcolor{lightblue}
  \textbf{Arbeitsschritt} & \textbf{Aufgabe im Detail} & \textbf{Deadline} \\ \hline
  \textbf{Implementierungsphase} & Erstellen einer Basisstruktur & \\ \hline
    & Datenbank & \\ \hline
    & Import von PDFs via pdf-extractor & \\ \hline
    & Satisfaction of search queries & \\ \hline
    & Simple ranking system & \\ \hline
    & Improve ranking system & \\ \hline
    & Implement spelling checks & \\ \hline
    & Improve Userinterface & \\ \hline \hline
  \textbf{Schreibphase} & Introduction & \\ \hline
    & Related Work & \\ \hline
    & Unsupervised document structure analysis & \\ \hline
    & Structured Text Retrieval & \\ \hline
    & STR with an Object-Oriented Database System & \\ \hline
    & Search queries & \\ \hline
    & Spelling Checks & \\ \hline
    & Implementierung & \\ \hline
    & Basisstruktur & \\ \hline
    & Aufbau der Datenbank & \\ \hline
    & Userinterface & \\ \hline
    & Satisfaction of searchqueries & \\ \hline
    & Ranking system & \\ \hline
    & Spelling checks & \\ \hline
    & Result & \\ \hline
    & Conclusion & \\ \hline
    & Korrektur & \\ \hline
\end{tabular}

\end{document}
