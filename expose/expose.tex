\documentclass[a4paper, 12pt]{scrartcl}


\usepackage[english]{babel} 
\usepackage[utf8]{inputenc}
\usepackage[T1]{fontenc}
\usepackage{ae,aecompl}
\usepackage{psfrag}
\usepackage[table]{xcolor}
\usepackage[automark]{scrpage2}
\usepackage[pdftex]{graphicx}
\pdfcompresslevel=9
\usepackage[pdftex=true,
    backref,
    pagebackref=false,
    colorlinks=true,
    bookmarks=true,
    bookmarksopen=false,
    bookmarksnumbered=false,
    pdfpagemode=None
  ]{hyperref}
\DeclareGraphicsExtensions{.pdf}

\hypersetup{
  pdftitle={}, %%
  pdfauthor={}, %%
  pdfsubject={}, %%
  pdfcreator={Accomplished with LaTeX2e and pdfLaTeX with hyperref-package.}, %% 
  pdfproducer={}, %%
  pdfkeywords={} %%
}

\title{The Use of Structured Text Retrieval to Search for Scientific Publications}

\author{Betreuer: \\
Dipl.-Ing. Dr.techn. Roman Kern \\ \\ \\ \\ \\
Thomas Mauerhofer (1031957)}

\date{Graz, am \today{}}

\bibliographystyle{plain}

\pagestyle{scrheadings}

\definecolor{lightblue}{rgb}{0.93,0.95,1.0}

\begin{document}

\maketitle
 
\newpage

\tableofcontents

\newpage

\section{Introduction}
\label{sec:Introduction}

In modern society searching for information via the Internet and specifically via Google's search engine has become a regular part of
day-to-day life. While Google registered 10,000 searches a day in 1998, the same number of queries was sent per second in 2006 \cite{Google}.
Therefore it stands to reason that the amount of accessible information is increasing steadily. On a daily basis, new websites are created,
articles are written and scientific papers are published. To manage this amount of data and in order to distinguish between relevant and non-relevant
sources various search engines are used. In this regard, searches should be simple and yet precise.

Looking at search engines for scientific publications and research there exists a broad selection of possibilities. One of the best known engines is
Google Scholar. It uses a simple input interface and lists the results with respect to their relevance. These listings contain data
in various formats from different years covering a variety of topics which do not necessarily reflect the search criteria. However, especially
while working on a scientific paper it is vital to obtain precise results.

This paper is concerned with improving the quality of search queries and their results for scientific publications in the Portable Document
Format (PDF). A common problem while searching for a specific author, for example, is that the most popular search engines often do not only list all of
his or her articles and books but also sources that cite these publications. The objective of this paper is to provide measures which enhance the stability of the search
terms and optimize the usability in order to deliver more concise results. To do so, simple search query structures, an intuitive front end, spell checks
and post-processing in the back end will be implemented.

\section{Related Work}

As stated in the \nameref{sec:Introduction} this intended master's thesis is concerned with improving the quality of search queries and their results for scientific
publications in PDF. According to \cite{KGJK14} all academic publications consist of similar structures. These structures are divided into chapters,
sections, subsections and so on. The Structured Text Retrieval Model, as proposed in \cite{RNBY99}, can be applied to these documents very well.
The model describes the handling of search queries and their corresponding results as well as the post-processing in the back end using structural meta information. 
In order to sort the documents with respect to their relevance, various ranking strategies such as the Jelenik-Mercer smoothing are used. Through extensions like the 
contextualization strategy and aggregation strategy the ranking will be refined further.

To ensure simple and stable search query structures the syntax will be constructed as described in \cite{Coh03}. Thereby the queries consist of multiple terms
which are structured according to \textit{label}:\textit{keyword}. Furthermore, the terms can be prefixed with \textit{+} to prioritize them specifically high. By means
of this structure the user interface will be extended with an advanced search in order to improve usability and therefore create an intuitive front end.

The final usability improvement is the achieved through spell checks. These can be implemented as described in \cite{SPCB13}. In it the Levenshtein Distance is used to verify 
that a word is spelled correctly.

\section{Structure}

At the beginning the fundamental structure for the project will be implemented. The micro framework Flask requires the front end to be coded in JavaScript and the back end in Python. 
Via simple requests inputs from the user side are sent to the server side. There the query is processed and a response is generated. These components form a solid ground work 
for the Text Retrieval Model.

Python provides a native library to connect to MySQL databases. It will be used to generate a database as proposed in \cite{YA94}. This database will
be filled with scientific publications via the tool described in \cite{KGJK14}. Then a ranking system, according to \cite{MRS08, RNBY99} can be implemented.

The next step is to apply spell checking to the search queries in order to enhance the usability.

Finally, an advanced search will be added that allows the user to formulate searches without having to know about the syntax of the queries, which will result in
an intuitive and simple user interface.

\section{Outline}

\begin{enumerate}
 \item Introduction
 \item Related Work
 \begin{enumerate}
  \item Unsupervised document structure analysis
  \item Structured Text Retrieval
  \item Structured-Text Retrieval System with an Object-Oriented Database System
  \item Search queries
  \item Real-word error detection and correction
 \end{enumerate}
 \item Implementation
 \begin{enumerate}
  \item Fundamental Structure
  \item Setup of the Database
  \item Userinterface
  \item Satisfaction of searchqueries
  \item Ranking system
  \item Spelling checks
 \end{enumerate}
 \item Results
 \item Conclusion
\end{enumerate}

\section{Selected Bibliography}

\bibliography{literature}
\nocite{*}

\section{Schedule}


\begin{tabular}{ | l | l | c | }
  \hline
  \rowcolor{lightblue}
  \textbf{Step} & \textbf{Task in Detail} & \textbf{Deadline} \\ \hline
  \textbf{Implementation Process} & Creation of the Fundamental Structure & \\ \hline
    & Database & \\ \hline
    & Import of PDFs via pdf-extractor & \\ \hline
    & Satisfaction of search queries & \\ \hline
    & Simple Ranking System & \\ \hline
    & Improved Ranking System & \\ \hline
    & Implement Spell Checks & \\ \hline
    & Improve Userinterface & \\ \hline \hline
  \textbf{Writing Process} & Introduction & \\ \hline
    & Related Work & \\ \hline
    & Unsupervised document structure analysis & \\ \hline
    & Structured Text Retrieval & \\ \hline
    & STR with an Object-Oriented Database System & \\ \hline
    & Search queries & \\ \hline
    & Spelling Checks & \\ \hline
    & Implementation & \\ \hline
    & Fundamental Structure & \\ \hline
    & Setup of the Database & \\ \hline
    & Userinterface & \\ \hline
    & Satisfaction of searchqueries & \\ \hline
    & Ranking system & \\ \hline
    & Spelling checks & \\ \hline
    & Results & \\ \hline
    & Conclusion & \\ \hline
    & Korrektur & \\ \hline
\end{tabular}

\end{document}
