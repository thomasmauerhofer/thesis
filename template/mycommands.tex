%doc%
%doc% \section{\texttt{mycommands.tex} --- various definitions}\myinteresting
%doc% \label{sec:mycommands}
%doc%
%doc% In file \verb#template/mycommands.tex# many useful commands are being
%doc% defined.
%doc%
%doc% \paragraph{What should I do with this file?} Please take a look at its
%doc% content to get the most out of your document.
%doc%

%doc%
%doc% One of the best advantages of \LaTeX{} compared to \myacro{WYSIWYG} software products is
%doc% the possibility to define and use macros within text. This empowers the user to
%doc% a great extend.  Many things can be defined using \verb#\newcommand{}# and
%doc% automates repeating tasks. It is recommended to use macros not only for
%doc% repetitive tasks but also for separating form from content such as \myacro{CSS}
%doc% does for \myacro{XHTML}. Think of including graphics in your document: after
%doc% writing your book, you might want to change all captions to the upper side of
%doc% each figure. In this case you either have to modify all
%doc% \texttt{includegraphics} commands or you were clever enough to define something
%doc% like \verb#\myfig#\footnote{See below for a detailed description}. Using a
%doc% macro for including graphics enables you to modify the position caption on only
%doc% \emph{one} place: at the definition of the macro.
%doc%
%doc% The following section describes some macros that came with this document template
%doc% from \myLaT and you are welcome to modify or extend them or to create
%doc% your own macros!
%doc%

%doc%
%doc% \subsection{\texttt{myfig} --- including graphics made easy}
%doc%
%doc% The classic: you can easily add graphics to your document with \verb#\myfig#:
%doc% \begin{verbatim}
%doc%  \myfig{flower}%% filename w/o extension in the folder figures
%doc%        {width=0.7\textwidth}%% maximum width/height, aspect ratio will be kept
%doc%        {This flower was photographed at my home town in 2010}%% caption
%doc%        {Home town flower}%% optional (short) caption for list of figures
%doc%        {fig:flower}%% label
%doc% \end{verbatim}
%doc%
%doc% There are many advantages of this command (compared to manual
%doc% \texttt{figure} environments and \texttt{includegraphics} commands:
%doc% \begin{itemize}
%doc% \item consistent style throughout the whole document
%doc% \item easy to change; for example move caption on top
%doc% \item much less characters to type (faster, error prone)
%doc% \item less visual clutter in the \TeX{}-files
%doc% \end{itemize}
%doc%
%doc%
\newcommand{\myfig}[5]{
%% example:
% \myfig{}%% filename in figures folder
%       {width=0.5\textwidth,height=0.5\textheight}%% maximum width/height, aspect ratio will be kept
%       {}%% caption
%       {}%% optional (short) caption for list of figures
%       {}%% label
\begin{figure}%[htp]
  \centering
  \includegraphics[keepaspectratio,#2]{figures/#1}
  \caption[#4]{#3}
  \label{#5} %% NOTE: always label *after* caption!
\end{figure}
}


%doc%
%doc% \subsection{\texttt{myclone} --- repeat things!}
%doc%
%doc% Using \verb#\myclone[42]{foobar}# results the text \enquote{foobar} printed 42 times.
%doc% But you can not only repeat text output with \texttt{myclone}.
%doc%
%doc% Default argument
%doc% for the optional parameter \enquote{number of times} (like \enquote{42} in the example above)
%doc% is set to two.
%doc%
%% \myclone[x]{text}
\newcounter{myclonecnt}
\newcommand{\myclone}[2][2]{%
  \setcounter{myclonecnt}{#1}%
  \whiledo{\value{myclonecnt}>0}{#2\addtocounter{myclonecnt}{-1}}%
}

\usepackage{array}
\newcolumntype{L}[1]{>{\raggedright\let\newline\\\arraybackslash\hspace{0pt}}m{#1}}
\newcolumntype{C}[1]{>{\centering\let\newline\\\arraybackslash\hspace{0pt}}m{#1}}
\newcolumntype{R}[1]{>{\raggedleft\let\newline\\\arraybackslash\hspace{0pt}}m{#1}}


%%%% End
%%% Local Variables:
%%% mode: latex
%%% mode: auto-fill
%%% mode: flyspell
%%% eval: (ispell-change-dictionary "en_US")
%%% TeX-master: "../main"
%%% End:
%% vim:foldmethod=expr
%% vim:fde=getline(v\:lnum)=~'^%%%%'?0\:getline(v\:lnum)=~'^%doc.*\ .\\%(sub\\)\\?section{.\\+'?'>1'\:'1':
